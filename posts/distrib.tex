
\documentclass{article}

\usepackage[utf8]{inputenc}
\usepackage[spanish]{babel}
\usepackage{graphicx}
\usepackage{xcolor}
\usepackage{amssymb}
\usepackage{amsthm}
\usepackage{amsfonts,amsmath,amsbsy,latexsym,mathrsfs}
\usepackage{lineno}
\usepackage{enumerate}
\usepackage{hyperref}

\title{Some important probability distributions}
\author{Manuel Sedano Mendoza}
\date{15 de Noviembre 2022}

\newtheorem*{ejem}{Ejemplo con calificación}
\newtheorem*{ejemp}{Ejemplo}
\newtheorem*{remark}{Cuidado}
\newtheorem*{thm}{Resultado}

\begin{document}

\maketitle
\begin{abstract}
In this entry, we will study some of the most famous probability distributions, and try to understand the random variables that follow such distributions.
\end{abstract}


Binomial ${n \choose x} p^x(1-p)^{n-x}$

\begin{thebibliography}{}
\bibitem{wa} \textsc{L. Wasserman}, \textit{All of Statistics: A concise course in Statistical Inference}. 


\end{thebibliography}

\end{document}
